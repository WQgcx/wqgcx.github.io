%
%  使用 XeLaTeX 编译
%
\documentclass[UTF8]{ctexart}
\usepackage[a4paper,top=0.5cm,bottom=0.5cm,left=1.0cm,right=1.05cm,%
            includehead,includefoot]{geometry}
\usepackage{graphicx}
\usepackage{enumerate}
\usepackage{float}
\usepackage{caption}
\usepackage{subfigure}
\usepackage[justification=centering]{caption}
\usepackage{appendix}

%%%%% ===== 设置章节
%\ctexset{%
%  section/name = {第,节},
%  section/number = \chinese{section},
%  section/format = \centering,
%  section/nameformat = \large\bfseries,
%  section/titleformat = \large\bfseries,
%  section/beforeskip = 5.5ex plus 1ex minus .2ex,
%
%  subsection/format=\raggedright,
%  subsection/nameformat=\bfseries,
%  subsection/titleformat=\bfseries
%}

%%%%%===== 宏包调用
\usepackage{amsmath,amssymb,amsfonts} % 数学宏包
\usepackage{bm} % 数学宏包
\usepackage{mathrsfs}
\usepackage{diagbox}
\usepackage{cases} % 数学宏包
\usepackage{yhmath} % 数学宏包
\usepackage{graphicx,subfigure} % 插图宏包
\usepackage{epstopdf} % eps转pdf宏包
\usepackage{xcolor,float} % 颜色与浮动对象宏包
\usepackage[breaklinks,colorlinks]{hyperref} % 超链接宏包
\hypersetup{citecolor=blue,         % 引用标记颜色
            linkcolor=blue,         % 内部普通链接的颜色
            urlcolor=blue,          % url 链接的颜色
            bookmarksnumbered=true, % 书签带章节编号
            bookmarksopen=true     % 书签目录展开
           }
\usepackage{booktabs} % 表格宏包
\usepackage{fancyvrb} % 摘录宏包
\usepackage{bbding}
  \fvset{formatcom=\color{blue},frame=single,rulecolor=\color{red}}
\usepackage[numbers,square,sort&compress]{natbib} % 参考文献
\renewcommand{\abstractname}{\large 摘要\\} % 摘要
\usepackage[version = 4]{mhchem}  % 化学反应

%%%%% ===== 定理环境
\usepackage[amsmath,thref,thmmarks,hyperref]{ntheorem} % 定理宏包
\theorempreskipamount1em % spacing before the environment
\theorempostskipamount0em  % spacing after the environment
\theoremstyle{plain}
\theoremheaderfont{\normalfont\heiti}
\theorembodyfont{\normalfont\kaishu}
\theoremindent0em
\theoremseparator{\hspace{0.2em}}
\theoremnumbering{arabic}
\newtheorem{theorem}{\textbf{Theorem}}[section]
\newtheorem{property}{\textbf{Property}}[section]
\newtheorem{definition}{\textbf{Definition}}[section]
\newtheorem{lemma}{\textbf{Lemma}}[section]
\newtheorem{remark}{\textbf{Remark}}[section]
\newtheorem{corollary}{\textbf{Corollary}}[section]
\newtheorem{example}{\textbf{Example}}[section]
%
\theoremstyle{nonumberplain}
\theorembodyfont{\normalfont}
\theoremsymbol{\ensuremath{\Box}}
\newtheorem{proof}{\small\emph{Proof}}


%===== 算法与源代码
\usepackage{algorithm,algpseudocode}  % 算法
\usepackage{listings} % 源代码
\renewcommand{\lstlistlistingname}{源代码目录}
\renewcommand{\lstlistingname}{源代码}
\lstset{language=Matlab}
\lstset{escapechar=`}
\lstset{basicstyle=\ttfamily\small,showstringspaces=false,tabsize=2}
\lstset{flexiblecolumns=true}
\lstset{xleftmargin=1ex,xrightmargin=1ex}
\lstset{frame=tblr,frameround=tttt}  %单线, 圆角框
%%\lstset{frame=TBLR}  %双线方框
%\lstset{frame=shadowbox,rulesepcolor=\color{blue}}
\lstset{commentstyle=\color{red},keywordstyle=\color{blue},caption=\lstname,%
        breaklines=true,backgroundcolor=\color{lightgray!20}}
\lstset{numbers=left, numberstyle=\small, stepnumber=1, numbersep=1em}

%%%%%===== 页眉页脚 =====================================================
\usepackage{fancyhdr}
\pagestyle{fancy}
\fancyhf{}
\renewcommand{\headrulewidth}{0pt}
\renewcommand{\sectionmark}[1]{\markboth{\uppercase{#1}}{}}
\chead{\leftmark}
\cfoot{\thepage}

%%%%% ===== 其他设置
\numberwithin{equation}{section} % 数学公式编号方式
\allowdisplaybreaks  % 允许在长公式中换页

%%%%% ===== 自定义命令
\newcommand{\CC}{\ensuremath{\mathbb{C}}}
\newcommand{\RR}{\ensuremath{\mathbb{R}}}
\newcommand{\A}{\mathcal{A}}
\newcommand{\lam}{\lambda}
\newcommand{\ii}{\bm{\mathrm{i}}\,} % 虚部
\newcommand{\NN}{\ensuremath{\mathbb{N}}}
\newcommand{\R}{\mathcal{R}}
\renewcommand{\L}{\mathcal{L}}
\renewcommand{\t}{\text}
\newcommand{\EE}{\mathbb{E}}
\newcommand{\scr}{\mathscr}
\renewcommand{\cal}{\mathcal}
\renewcommand{\P}{\mathbb{P}}
\newcommand{\Q}{\mathbb{Q}}
\renewcommand{\d}{\text{d}}
%%%%% ===== 论文开始 =====================================================
\begin{document}
\title{\Huge Stochastic Processes}

\author{Lectured by \href{mailto:weijunxu@bicmr.pku.edu.cn}{Weijun Xu}\qquad\qquad\LaTeX ed by \href{https://wqgcx.github.io/}{Chengxin Gong}}

\maketitle % 生成标题

\tableofcontents

\newpage
\section{Review of Martingales}
\begin{itemize}
  \item $(X_n)_{n\geq 0}$ is $L^2$-bounded martingale $\Rightarrow X_n$ converges in $L^2$.
  \item $(X_n)_{n\geq 0}$ is $L^1$-bounded martingale $\Rightarrow X_n$ converges a.s.
  \item (1) + (2): If $(X_n)_{n\geq 0}$ is $L^p$-bounded martingale for $p>1$, then $X_n$ converges in $L^{p'}$ for $p'\in[1,p)$.
  \item Statement is false when $p=1$. Example: $\Omega=[0,1),\scr{F}_n=\sigma\{[\frac{i}{2^n},\frac{i+1}{2^n})\}_{i=0}^{2^n-1},X_n(\omega):=\begin{cases}
    2^n&\omega\in[0,\frac{1}{2^n})\\0&\t{otherwise}
  \end{cases}$.
  \item Let $p>1$ and $(X_n)_{n\geq 0}$ be $L^p$ bounded martingale w.r.t. $\scr{F}_n$. Then $\exists X\in L^p(\Omega,\scr{F}_{\infty},P)$ s.t. $X_n\to X$ in $L^p$ and a.s. and $X_n=\EE(X|\scr{F}_n)$.
  \item Doob's maximal inequality: Let $p>1,\exists C=C_p$ s.t. $\forall$ martingale $(X_n)_{n\geq 0}$, we have $\EE|X_n^*|^p\leq C_p\EE|X_n|^p$ where $|X_n^*|=\sup_{0\leq k\leq n}\sup|X_k|$.
  \item Let $(Z_n)_{n\geq 0}$ be a nonnegative sub-martingale and $Z_n^*=\sup_{0\leq k\leq n}Z_k$, then $\P(Z_n^*>\lambda)\leq\frac{1}{\lambda}\EE(Z_n1_{\{Z_n^*>\lambda\}})\leq\frac{1}{\lambda}\EE Z_n$. Corollary: $\P(Z_n^*>\lambda)\leq\frac{1}{\lambda^p}\EE(Z_n^p1_{\{Z_n^*>\lambda\}})\leq\frac{1}{\lambda^p}\EE(Z_n^p)$.
  \item If $(X_n)_{n\geq 0}$ is a martingale with $\sup_n\EE(|X_n|\log(1+|X_n|))<+\infty$, then $X_n$ converges in $L^1$.
  \item Two probability measures $\P$ and $\Q$ on $(\Omega,\scr{F})$, $\Q<<\P$ on $\scr{F}_n$ for every $n$ and $M_n=\frac{\d\Q|_{\scr{F}_n}}{\d\P|_{\scr{F}_n}}$. $(M_n)_{n\geq 0}$ is a $\P$-martingale w.r.t. $(\scr{F}_n)_{n\geq 0}$. $\Q<<\P$ on $\scr{F}_\infty$ if and only if $M_n\to M$ in $L^1$. $\Q(A)=\int_AM\d\P+\Q(A\cap\{M=+\infty\})$.
  \begin{proof}
    \small\emph{Sufficiency. $\Q<<\P$ on $\scr{F}=\scr{F}_\infty$, thus let $Z=\frac{\d\Q|_{\scr{F}}}{\d\P|_{\scr{F}}}$, we need to show $M_n$ converges to $Z$ in $L^1$. $\forall A\in\scr{F}_n,\int_AM_n\d\P=\Q(A)=\int_AZ\d\P\Rightarrow M_n=\EE(Z|\scr{F}_n)$. Thus $M_n$ is uniformly integrable, thus converges in $L^1$.\\
    Necessity. Suppose $M_n\to M$ a.s. and in $L^1$ We need to show $M_n=\EE (M|\scr{F}_n)$ and $M=\frac{\d\Q}{\d\P}$. It suffices to show $\Q(A)=\int_AM\d\P$ for all $A\in\cup_n\scr{F}_n$. Suppose $A\in\scr{F}_N$. Then $\Q(A)=\int_AM_N\d\P=\int_AM_{N+k}\d\P\to \int_AM\d\P$. By $\pi-\lambda$ theorem we obtain the result.\\
    Suppose $\P\perp\Q$ on $\scr{F}$(i.e. $\exists E$ s.t. $\P(E)=1,\Q(E^c)=1$) and $\P<<\Q$ on $\scr{F}_n$. Then $\frac{1}{M_n}$ converges $\Q$-a.s. Let $\mathbb{R}=\frac{1}{2}(\P+\Q)$, $\P,\Q<<\mathbb{R}$ on $\scr{F}$, $\frac{\d\P|_{\scr{F}_n}}{\d\mathbb{R}|_{\scr{F}_n}}=\frac{2}{1+M_n}\to\frac{2M}{1+M}$ in $L^1(\RR)$, $\frac{\d\Q}{\d\mathbb{R}}=\frac{2M_n}{1+M_n}\to\frac{2}{1+M}$ in $L^1(\RR)$. Then $\Q(A)=\Q(A\cap E^c)=\int_{A\cap E^c}\frac{2M}{1+M}\d\RR=\int_A\frac{2M}{1+M}1_{E^c}\d\RR\stackrel{\P(E^c)=0}{=}2\RR(A\cap E^c)=2\int_A1_{E^c}\d\RR\Rightarrow M=+\infty$ on $E^c\Rightarrow \Q(M=+\infty)=1$. Similarly $\P(M=0)=\Q(M=+\infty)=1$.\\
    General situation: $\Q=\Q_1+\Q_2,\Q_1<<\P,\Q_2\perp\P$ on $\scr{F}$. Then we can write $M_n=Y_n+Z_n$ where $Y_n\to Y$ in $L^1(\P)$ and $Z_n\to 0$ $\P$-a.s. $\Q_1(A)=\int_AY\d\P=\int_AM\d\P$. $\Q_2(A)=\Q_2(A\cap\{Z=+\infty\})$. Since $Z=0$ $\P$-a.s., $M<+\infty$ $\P$-a.s. and $\Q_2(M=+\infty)=1$, we have $\Q_2(A)=\Q(A\cap\{Z=+\infty\})=\Q_2(A\cap\{M=+\infty\})=\Q(A\cap\{M=+\infty\})$. To sum up, $\Q(A)=\int_AM\d\P+\Q(A\cap\{M=+\infty\})$.}
  \end{proof}
  \item Statement is false if $M_n\not\to M$ in $L^1$. Example: $\Omega=\{\omega=(\omega_1,\cdots,\omega_n,\cdots)\in\{\pm 1\}^{\mathbb{N}}\},X_n(\omega)=\omega_n$. $X_n$'s are i.i.d. under $\P$ and $\Q$, but $\P(X_n=1)=\frac{1}{2},\P(X_n=-1)=\frac{1}{2},\Q(X_n=1)=\frac{1}{3},\Q(X_n=-1)=\frac{2}{3}$. $\scr{F}_n=\sigma(X_1,\cdots,X_n)$. $\P(\lim_{n\to\infty}\frac{1}{n}\sum_{k=1}^nX_k=0)=1,\Q(\lim_{n\to\infty}\frac{1}{n}\sum_{k=1}^nX_k=-\frac{1}{3})=1$.
  \item Monotone class theorem for functions: Suppose $\cal{A}$ us a $\pi$-system and $\cal{H}$ be a class of functions from $\Omega$ to $\RR$ s.t. (1) $1_A\in\cal{H}$ for every $A\in\scr{A}$, (2) if $f,g\in\cal{H}$ then $af+bg\in\cal{H}$, (3) if $f_n\in\cal{H}$ and $f_n\uparrow f$ then $f\in\cal{H}$. Then all nonnegative $\sigma(\cal{A})$-measurable functions are in $\cal{H}$.
  \item Let $(Y_n)_{n\geq 0}$ be i.i.d., nonnegative r.v.'s with $\EE Y_k=1$. Then $M_n=\prod_{k=1}^nY_k$ converges in $L^1$ iff $Y_n\equiv 1$. Otherwise $M_n\to 0$ a.s.
  \begin{proof}\small\emph{Note that $\frac{1}{n}\log M_n=\frac{1}{n}\sum_{k=1}^n\log Y_k\to\EE\log Y$ a.s. If $\EE\log Y=0$ then by Jensen's inequality we have $Y_n\equiv 1$ which means $M_n$ converges in $L^1$. If $\EE\log Y<0$ then $M_n\to 0$ a.s.}
  \end{proof}
  \item Kakutani's theorem: $M_n=\prod_{k=1}^nY_k$, $Y_k\geq 0$ are independent, $\EE Y_k=1$, $\lambda_k=\EE\sqrt{Y_k}$. (1) If $\prod_k\lambda_k>0$, then $M_n\to M$ in $L^1$; (2) If $\prod_k\lambda_k=0$, then $M_n\to 0$ a.s.
  \begin{proof}\small\emph{
    Let $Z_n=\prod_{k=1}^n\frac{\sqrt{Y_k}}{\lambda_k}$. Then $Z_n$ is a martingale and has an a.s. limit $Z$, and $M_n=(\prod_{k=1}^n\lambda_k)^2Z_n^2$. If $\prod_k\lambda_k>0$, then $Z_n$ is $L^2$ bounded and then convergence in $L^2$, which implies $M_n\to M$ in $L^1$. If $\prod_k\lambda_k=0$, it is obvious that $M_n\to 0$ a.s.}
  \end{proof}
  \item Martingale LLN: Let $(M_n)_{n\geq 0}$ be a martingale s.t. $\sum_{k=1}^{+\infty}\frac{\EE(M_k-M_{k-1})^2}{k^2}<+\infty$. Then $\frac{M_n}{n}\to 0$ a.s.
  \begin{proof}\small\emph{
    Let $Y_n=\sum_{k=1}^n\frac{X_k}{k}$. Then $(Y_n)_{n\geq 0}$ is an $L^2$ bounded martingale, thus $Y_n\to Y$ a.s. Then by Kronecker's lemma, $M_n=\frac{X_1+\cdots+X_n}{n}\to 0$ a.s.
  }
  \end{proof}
  \item Martingale CLT: Let $(M_n)_{n\geq 0}$ be a martingale with $M_0=0$ and $\sigma_n^2=\sum_{k=1}^n\EE X_k^2=\EE\langle M\rangle_n$. Assume that $\frac{1}{\sigma_n^2}\max_{1\leq k\leq n}(\EE X_k^2)\to 0$, $\frac{1}{\sigma_n^2}\sum_{k=1}^n\EE(X_k^21_{\{|X_k|>\epsilon\sigma_n\}}|\scr{F}_{k-1})\stackrel{p}{\to}0$ for all $\epsilon>0$, $\frac{1}{\sigma_n^2}\langle M\rangle_n\stackrel{p}{\to} 1$. Then $\frac{M_n}{\sigma_n}\Rightarrow\mathcal{N}(0,1)$.
\end{itemize}
\section{Markov Chains}
\begin{itemize}
  \item Let $(X_n)_{n\geq 0}$ be a homogeneous Markov chain on a discrete space $S$. $\P^x:$ law of $(X_n)_{n\geq 0}$ conditioned on $X_0=x$. $\P(X_{n+1}\in A|\scr{F}_n)=\P^{X_n}(X_1\in A)=\P(X_1\in A|X_0=X_n)$. $\EE^x:$ expectation under $\P^x$. $\P^x(X_1=y)=p(x,y)$.
  \item For every $f:S\to\RR$ bounded, define $(\cal{P}f)(x)=\sum_{y\in S}p(x,y)f(y)=\EE^x(f(X_1))$, $(\mathcal{L}f)(x)=\sum_{y\in S}p(x,y)f(y)-f(x)$. $\mathcal{L}=\cal{P}-\t{id}$, the generator.
  \item Let $(X_n)_{n\geq 0}$ be a homogeneous Markov chain with generator $\mathcal{L}$. Then for every bounded $f:S\to\RR$, $M_n=f(X_n)-f(X_0)-\sum_{k=0}^{n-1}(\mathcal{L}f)(X_k)$ is a martingale. Conversely, let $(X_n)_{n\geq 0}$ be a process and $\mathcal{L}$ be an operator on $\cal{B}(S)$ s.t. $M_n^f$ is a martingale for every $f$, then $(X_n)_{n\geq 0}$ is a Markov chain with generator $\mathcal{L}$.
  \item Given operator $\mathcal{L}$ on $\cal{B}(S)$, we say $f:S\to\RR$ is (1) harmonic for $\mathcal{L}$ if $\mathcal{L}f=0$; (2) sub-harmonic for $\mathcal{L}$ if $\mathcal{L}f\geq 0$; (3) super-harmonic for $\mathcal{L}$ if $\mathcal{L}f\leq 0$.
  \item Let $f$ be the generator of a Markov chain $(X_n)_{n\geq 0}$. Then $f$ is (sub-/super-)harmonic $\Leftrightarrow$ $f(X_n)_{n\geq 0}$ is a (sub-/super-) martingale.
  \item $f$ is (sub-/super-)harmonic on $D\subset S$ if $\L f \geq/\leq/=0$ on $D$. Let $\tau=\inf\{k\geq 0:X_k\in D^c\}$, then $(f(X_{n\wedge\tau}))_{n\geq 0}$ is a (sub-/super)martingale.
  \item Maximum principle: Let $(X_n)_{n\geq 0}$ be a Markov chain and $D\subset S$ s.t. the stopping time $\tau=\inf\{k\geq 0,X_k\in D^c\}$ is a.s. finite. If $f$ is bounded and sub-harmonic on $D$, then $\sup_{x\in D}f(x)\leq\sup_{x\in D^c}f(x)$.
  \begin{proof}
    \small\emph{$f$ is sub-harmonic implies $(f(X_{n\wedge\tau}))$ is a sub-martingale, hence for $x\in D$ we have $f(x)\leq\EE^xf(X_{n\wedge\tau}|)\to\EE^x(f(X_\tau))\leq\sup_{x\in D^c}f(x)$.}
  \end{proof}
  \item $A\subset S,\tau_A=\sup\{k\geq 0:X_k\in A\}$. (1) $u(x)=\P^x(\tau_A<+\infty)\Rightarrow \begin{cases}
    \mathcal{L}u=0&\t{on }A^c\\u=1&\t{on }A
  \end{cases}$. (2) $u(x)=\P(\tau_A<\tau_B)\Rightarrow\begin{cases}
    \cal{L}u=0&\t{on } (A\cup B)^c\\u=1&\t{on }A\\u=0&\t{on }B.
  \end{cases}$. (3) $u(x)=\EE^x[\tau_A]\Rightarrow\begin{cases}
    \cal{L}u=-1&\t{on }A^c\\u=0&\t{on }A
  \end{cases}$.
  \item Any nonnegative solution $v$ to $\begin{cases}
    \mathcal{L}v=0&\t{on }A^c\\v=1&\t{on }A
  \end{cases}$ satisfies $v\geq u$. Furthermore, if $u\equiv 1$, then $\exists 1$ bounded solution to $\begin{cases}
    \mathcal{L}v=0&\t{on }A^c\\v=f&\t{on }A
  \end{cases}$ with $v(x)=\EE^x(f(X_{\tau_A}))$.
  \begin{proof}
    \small\emph{
      Let $v(x)$ be a non-negative solution, then $v(X_{n\wedge\tau_A})_{n\geq 0}$ is a martingale. $v(x)=\EE^x v(X_{n\wedge\tau_A})=\EE^xv(X_{n\wedge\tau_A})1_{\tau_A<\infty}+\EE^xv(X_{n\wedge\tau_A})1_{\tau_A=\infty}\geq\EE v(X_{n\wedge\tau_A})1_{\tau_A<\infty}$. Let $n\to\infty$ and by Fatou's lemma, we have $v(x)\geq \EE^x v(X_{\tau_A})1_{\tau_A<\infty}=\P^x(\tau_A<\infty)=u(x)$. If $u(x)\equiv 1$ and $v(x)$ is bounded, then by bounded convergence theorem, $v(x)=\EE^xv(X_{n\wedge\tau_A})\to\EE^x v(X_{\tau_A})=\EE^x f(X_{\tau_A})$.
    }
  \end{proof}
  \item Doob's $h$-transform: Let $h$ be nonnegative, harmonic with $h(x_0)=1$ for some $x_0\in S$. Then $(h(X_n))_{n\geq 0}$ is a martingale with $\EE^{\P^{x_0}}(h(X_n))=1$. Then $\exists 1$ measure $\Q^h$ on $\scr{F}_\infty$ s.t. $\frac{\d\Q^h}{\d\P^{x_0}|_{\scr{F}_n}}=h(X_n),\forall n\geq 0$. $\Q^h(X_0=x_0)=1$, $(X_n)_{n\geq 0}$ never visits the set $D=\{x:h(x)=0\}$. Under $\Q^h$, $(X_n)_{n\geq 0}$ is again a Markov chain on $S\backslash D$ with transition probability $q(x,y)=\frac{p(x,y)h(y)}{h(x)}$ (or equivalently, $(\mathcal{L}^hf)(x)=\frac{1}{h(x)}(\mathcal{L}(hf))(x)$).
  \begin{proof}
    \small\emph{The first two statements are obvious. Then by definition, we have 
      $\Q(X_{n+1}=y|\scr{F}_n)=\frac{\Q(X_{n+1}=y,X_n=x_n,\cdots,X_0=x_0)}{\Q(X_n=x_n,\cdots,X_0=x_0)}=\frac{\int_{\{X_{n+1}=y,X_n=x_n,\cdots,X_0=x_0\}}h(X_{n+1})\d\P^{x_0}}{\int_{\{X_n=x_n,\cdots,X_0=x_0\}}h(X_n)\d\P^{x_0}}=\frac{h(y)\P^{x_0}(X_{n+1}=y,X_n=x_n,\cdots,X_0=x_0)}{h(x_n)\P^{x_0}(X_n=x_n,\cdots,X_0=x_0)}=\frac{h(y)p(x_n,y)}{h(x_n)}$. Next we show $M_n^f:=f(X_n)-f(X_0)-\sum_{k=0}^{n-1}(\cal{L}^hf)(X_k)$ is a $\Q$-martingale for any bounded $f$. Let $Z_n=\EE^{\Q}f(X_{n+1})|\scr{F}_n$. $\forall A\in\scr{F}_n$, $\int_AZ_nh(X_n)\d\P^{x_0}=\int_AZ_n\d\Q=\int_Af(X_{n+1})\d\Q=\int_Af(X_{n+1})h(X_{n+1})\d\P^{x_0}=\EE^{\P^{x_0}}[\EE^{\P^{x_0}}(f(X_{n+1})h(X_{n+1})1_A|\scr{F}_n)]=\EE^{\P^{x_0}}[1_A\EE^{\P^{x_0}}(f(X_{n+1})h(X_{n+1})|\scr{F}_n)]=\int_A\cal{P}(hf)(X_n)\d\P^{x_0}$. Thus $Z_n=\frac{\cal{P}(hf)(X_n)}{h(X_n)}$ only depends on $X_n$, hence $(X_n)_{n\geq 0}$ is a Markov chain on $\Q$ with generator $\cal{L}^h$.
    }
  \end{proof}
  \item An irreducible Markov chain $(X_n)_{n\geq 0}$ (1) is transient if $\exists x$ and $A\subset S$ s.t. $\P(\tau_A<\infty|X_0=x)<1$; (2) is recurrent if $\exists$ a finite set $A\subset S$ s.t. $\P(\tau_A<\infty)=1$ for all $x\in S$. (3) is positive recurrent if $\exists$ a finite set $A\subset S$ s.t. $\EE(\tau_A)<\infty$ for all $x\in S$.
  \item Foster-Lyapunov criterion: An irreducible MC on a countable state space $S$ (1) is transient iff $\exists v:S\to\RR^+$ and $A\subset S$ non-empty s.t. $\mathcal{L}v\leq 0$ on $A^c$ and $v(x)<\inf_{y\in A}v(y)$ for some $x\in A^c$; (2) is recurrent iff $\exists v:S\to\RR^+$ s.t. $\mathcal{L}v\leq 0$ on $A^c$ where $A$ is a finite set and $\{x:v(x)\leq N\}$ is finite for every $N$; (3) is positive recurrent iff $\exists v:S\to\RR^+$, $A\subset S$ finite, $\exists\epsilon>0$ s.t. $\mathcal{L}v\leq-\epsilon$ on $A^c$ and $\sum_{y\in S}p(x,y)V(y)<+\infty$ for all $x\in A$.
  \begin{proof}
    \small\emph{(1) $v(X_{n\wedge\tau_A})_{n\geq 0}$ is a super-martingale, hence $v(x)\geq\EE v(X_{n\wedge\tau_A}\geq\EE v(X_{n\wedge\tau_A})1_{\tau_A<\infty})$. Let $n\to\infty$ we know $v(x)\geq\EE v(X_{\tau_A}1_{\tau_A<\infty})\geq(\inf_{y\in A}v(y))\P
    ^x(\tau_A<\infty)\Rightarrow \P^x(\tau_A<\infty)<\frac{v(x)}{\inf_{y\in A}v(y)}<1$. (2) On $\{\tau_A=\infty\}$, $\lim\sup_{n\to\infty}v(X_{n\wedge\tau_A})=+\infty$ a.s. Since $(v(X_{n\wedge\tau_A}))_{n\geq 0}$ is a nonnegative super-martingale, hence converges a.s., therefore $\lim_{n\to\infty}v(X_{n\wedge\tau_A})=+\infty$ a.s. Note that $v(x)\geq\EE v(X_{n\wedge\tau_A})1_{\tau_A=\infty}$. Since LHS is a finite number, we have $\P^x(\tau_A=\infty)=0$. (3) $\EE v(X_{n\wedge\tau_A})|\scr{F}_{n-1}\leq v(X_{(n-1)\wedge\tau_A})-\epsilon 1_{\tau_A\geq n}$. Taking expectation on the both sides, $\EE v(X_{n\wedge\tau_A})\leq \EE v(X_{(n-1)\wedge\tau_A})-\epsilon\EE^x1_{\tau_A\geq n}\leq\cdots\leq v(x)-\epsilon\sum_{k=1}^n\P^x(\tau_A\geq k)\Rightarrow\EE^x\tau_A=\sum_{k=1}^{\infty}\P^x(\tau_A\geq k)\leq\frac{v(x)}{\epsilon}<\infty$.}
  \end{proof}
  \item e.g. $h(x)=\frac{\P^x(\tau_A<\tau_B)}{\P^{x_0}(\tau_A<\tau_B)}$ is harmonic on $(A\cup B)^c$ with $h(x_0)=1 (x_0\in(A\cup B)^c)$. Then $\forall x,y\in(A\cup B)^c,q(x,y)=\frac{h(y)p(x,y)}{h(x)}=\frac{\P^y(\tau_A<\tau_B)p(x,y)}{\P^x(\tau_A<\tau_B)}=\frac{\P^x(X_1=y,\tau_A<\tau_B)}{\P^x(\tau_A<\tau_B)}=\P^x(X_1=y|\tau_A<\tau_B)$.
  \item e.g. $\P$ is simple symmetric random walk on $\mathbb{Z}$ starting from $X_0=0$. Question: what is the law of $(X_n)_{n\geq 0}$ conditioned on $X_n\geq 0$ for all $n$? Let $\tau_k=\inf\{n\geq 0,X_n=k\}$. On $\{\tau_N<\tau_{-1}\},\frac{h(y)}{h(x)}=\frac{\P^y(\tau_N<\tau_{-1})}{\P^x(\tau_N<\tau_{-1})}=\frac{y+1}{x+1}$. Thus $q_N(x,y)=\frac{1}{2}\frac{y+1}{x+1},|x-y|=1,x\in\{0,\cdots,N-1\}\Rightarrow q(x,y)=\frac{1}{2}\frac{y+1}{x+1},x\geq 0,|x-y|=1$.
\end{itemize}
\section{Ergodic Theorem}
\begin{itemize}
  \item 
\end{itemize}
\end{document}